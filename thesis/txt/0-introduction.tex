
\providecommand{\myrootdir}{..}
\documentclass[\myrootdir/main.tex]{subfiles}

\begin{document}

\chapter{Introduction}
\plan{motivate why information extraction from build logs}
Many software projects now use continious integration (CI) to improve \mention{reasons to do this, look into Proksch papers}. These CI builds often produce very long and verbose build logs \mention{add log characteristics, cite what Moritz cited in his proposal?}, stating the progress and results of the various steps within the build.

These build logs are a highly valuable data source. First of all, for the developers that read them to analyze why their build failed or .. \mention{find more reasons}. Second, for researchers that can harvest the information contained in the logs -- and their metadata -- to study the software engineering process of a project. However they can only utilize the information within the build logs if they can adequately extract the information relevant to them.

There are many different extraction techniques use for this task. Beller et al.  used regular expressions when analysing the failure reasons of ruby and Java Maven buildlogs from TravisCI~\cite{beller2017oops}, while Vassallo et al. wrote a custom parser for Java Maven build logs to gather information for build repair hints~\cite{vassallo2018un-break}. Recently Amar et al. greatly reduced the number of build log lines for a developer to inspect by creating a diff between the logs from a failed and successful build~\cite{amar2019mining}. \bp{anecdotal: keyword search?}. Apart from those there are various more extraction techniques like searching for keywords, ... \todo{more?} \bp{introduce problem: which one to choose}

With our work we want to support developers, researchers or project managers in deciding which technique is the best one for their use case. 

\plan{what we do in this thesis:}

We aim to define a model to characterize the different extraction techniques and answer what influences the suitability of a technique. We support our assumption by evaluating three chosen techniques, namely regular expression synthesis by example using the Microsoft PROSE library, a common text similarity approach and simple keyword search. Our \emph{Failing Build Logs Data Set} encompasses about 800 log files from 80 repositories, labeled with the log part describing the reason a build failed, keywords to search for this extractions and a categorization of the extractions according to their structural position within the log \todo{is this understandable here? leave out?}. 

We aim to answer the following research questions:

\begin{itemize}
  \item[\textbf{RQ1:}] What criteria influence the suitability of an information extraction technique for CI build logs?
  \item[\textbf{RQ2:}] Is Programming by Example suited to extract information from CI build logs?
  \item[\textbf{RQ3:}] When are text similarity or keyword search better suited for information extraction from CI build logs?
\end{itemize}

Our study of the three techniques on \emph{Failing Build Logs Data Set} shows that PROSE is suitable for extraction tasks requiring a high precision, ... . Extracting information by using text similarity of examples is suitable when ... . Simple keyword search is suitable ... .

\plan{contributions:}

Our work contributes:

\begin{itemize}
  \item A model to characterize information extraction techniques from \todo{CI?} build logs
  \item A model to characterize use case scenarios for information extraction from CI build logs
  \item A model of the extractable information in CI build logs
  \item A tool unifying several information extraction techniques namely:
        \begin{itemize}
          \item our implementation of regular expression program synthesis using the Microsoft PROSE library,
          \item a common information retrieval approach using text similarity, and
          \item a simple keyword search approach
        \end{itemize}
  \item The \emph{LogCollector}, a tool to gather logs from Travis CI
  \item A validated data set of about 800 logs from failed Travis CI builds labeled with:
        \begin{itemize}
          \item the substring of the log describing the reason the build failed,
          \item keywords developers would use to search for these descriptions of the build failure reason, and
          \item a categorization of the extractions according to their structural position within the build log
        \end{itemize}
  \item An evaluation of our model for the three implemented information extraction techniques \todo{describe \emph{what} we look for with this evaluation}
  \item A extendable sceme supporting decision on (our three) information extraction techniques relative to a given use case scenario
\end{itemize}

\plan{structure overview:}

This thesis first presents an overview over the related research, spanning from CI research, build log analysis and augmentation, and common \todo{production?} log processing over information extraction and retrieval techniques to program synthesis by examples.
Next, chapter \ref{models} present our models for information extraction techniques, use case scenarios for such techniques and extractable information from CI build logs. It also introduces the three techniques, we focus on during our research evaluation, as well as their theoretical foundations.
Chapter \ref{data-set} describes the creation of our \emph{The Failing Build Logs Data Set} collected from failed Travis CI build logs, and the labeling and validation process.
Following, Chapter \ref{implementation} our implementation of the three chosen information extraction techniques and the usage of our unified information extraction tool.
This tool is also used for the evaluation study of our models described in Chapter \ref{evaluation}. There we also describe the resulting instantiation of our recommendation model for regex program synthesis, text similarity and keyword search.
Lastly, we conclude and give an overview of further research opportunities in Chapter \ref{conclusion}.
\end{document}
