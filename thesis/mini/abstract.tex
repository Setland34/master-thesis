\providecommand{\myrootdir}{..}
\documentclass[\myrootdir/main.tex]{subfiles}

\begin{document}

\chapter*{\myAbstractTitle}

Software builds from continuous integration produce detailed logs about the status and results of the various tools involved in the build.
These build logs are a valuable data source for developers and researchers to inspect test results, the duration of build steps or understand the cause of a build failure.
However, build logs are very verbose, at best semi-structured and their structure differs highly between projects.
This makes it hard to process and analyze them.
In this paper, we evaluate and compare three different techniques that aim to retrieve specified text chunks from a build log, namely program synthesis by example, textual similarity and search keywords.
We conduct an empirical study by comparing these techniques on our \emph{LogChunks} data set of 797 Travis CI logs from a diverse range of projects.
Our findings show that none of the three techniques clearly outperforms the others.
We discuss under which circumstances each technique performs best and provide a recommendation on when developers or researchers should use which technique.

\end{document}
