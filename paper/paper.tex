%%
%% This is file `sample-sigconf.tex',
%% generated with the docstrip utility.
%%
%% The original source files were:
%%
%% samples.dtx  (with options: `sigconf')
%% 
%% IMPORTANT NOTICE:
%% 
%% For the copyright see the source file.
%% 
%% Any modified versions of this file must be renamed
%% with new filenames distinct from sample-sigconf.tex.
%% 
%% For distribution of the original source see the terms
%% for copying and modification in the file samples.dtx.
%% 
%% This generated file may be distributed as long as the
%% original source files, as listed above, are part of the
%% same distribution. (The sources need not necessarily be
%% in the same archive or directory.)
%%
%% The first command in your LaTeX source must be the \documentclass command.
\documentclass[sigconf]{acmart}

% TODO remove XD
\usepackage{xcolor}
\newcommand{\secfunc}[1]{{\color{magenta}#1}}
\newcommand{\mention}[1]{{\color{cyan}#1}}
\newcommand{\plan}[1]{{\color{purple}#1}}
\newcommand{\bp}[1]{{\color{violet}#1}}
\newcommand{\draft}[1]{{\color{blue}#1}}
\newcommand{\review}[1]{{\color{black}#1}}
\newcommand{\todo}[1]{{\color{orange}#1}}
%%
%% \BibTeX command to typeset BibTeX logo in the docs
\AtBeginDocument{%
  \providecommand\BibTeX{{%
    \normalfont B\kern-0.5em{\scshape i\kern-0.25em b}\kern-0.8em\TeX}}}

%% Rights management information.  This information is sent to you
%% when you complete the rights form.  These commands have SAMPLE
%% values in them; it is your responsibility as an author to replace
%% the commands and values with those provided to you when you
%% complete the rights form.
\setcopyright{acmcopyright}
\copyrightyear{2018}
\acmYear{2018}
\acmDOI{10.1145/1122445.1122456}

%% These commands are for a PROCEEDINGS abstract or paper.
\acmConference[Woodstock '18]{Woodstock '18: ACM Symposium on Neural
  Gaze Detection}{June 03--05, 2018}{Woodstock, NY}
\acmBooktitle{Woodstock '18: ACM Symposium on Neural Gaze Detection,
  June 03--05, 2018, Woodstock, NY}
\acmPrice{15.00}
\acmISBN{978-1-4503-9999-9/18/06}


%%
%% Submission ID.
%% Use this when submitting an article to a sponsored event. You'll
%% receive a unique submission ID from the organizers
%% of the event, and this ID should be used as the parameter to this command.
%%\acmSubmissionID{123-A56-BU3}

%%
%% The majority of ACM publications use numbered citations and
%% references.  The command \citestyle{authoryear} switches to the
%% "author year" style.
%%
%% If you are preparing content for an event
%% sponsored by ACM SIGGRAPH, you must use the "author year" style of
%% citations and references.
%% Uncommenting
%% the next command will enable that style.
%%\citestyle{acmauthoryear}

%%
%% end of the preamble, start of the body of the document source.
\begin{document}

%%
%% The "title" command has an optional parameter,
%% allowing the author to define a "short title" to be used in page headers.
\title{Analyzing Buildlogs Using Programming by Example}

\author{Carolin Brandt}

%%
%% By default, the full list of authors will be used in the page
%% headers. Often, this list is too long, and will overlap
%% other information printed in the page headers. This command allows
%% the author to define a more concise list
%% of authors' names for this purpose.
\renewcommand{\shortauthors}{Brandt, et al.}

%%
%% The abstract is a short summary of the work to be presented in the
%% article.
\begin{abstract}
  ...
\end{abstract}

%%
%% The code below is generated by the tool at http://dl.acm.org/ccs.cfm.
%% Please copy and paste the code instead of the example below.
%%
%\begin{CCSXML}
%<ccs2012>
% <concept>
%  <concept_id>10010520.10010553.10010562</concept_id>
%  <concept_desc>Computer systems organization~Embedded systems</concept_desc>
%  <concept_significance>500</concept_significance>
% </concept>
% <concept>
%  <concept_id>10010520.10010575.10010755</concept_id>
%  <concept_desc>Computer systems organization~Redundancy</concept_desc>
%  <concept_significance>300</concept_significance>
% </concept>
% <concept>
%  <concept_id>10010520.10010553.10010554</concept_id>
%  <concept_desc>Computer systems organization~Robotics</concept_desc>
%  <concept_significance>100</concept_significance>
% </concept>
% <concept>
%  <concept_id>10003033.10003083.10003095</concept_id>
%  <concept_desc>Networks~Network reliability</concept_desc>
%  <concept_significance>100</concept_significance>
% </concept>
%</ccs2012>
%\end{CCSXML}
%
%\ccsdesc[500]{Computer systems organization~Embedded systems}
%\ccsdesc[300]{Computer systems organization~Redundancy}
%\ccsdesc{Computer systems organization~Robotics}
%\ccsdesc[100]{Networks~Network reliability}

%%
%% Keywords. The author(s) should pick words that accurately describe
%% the work being presented. Separate the keywords with commas.
\keywords{ci, buildlogs, programming by example}

%%
%% This command processes the author and affiliation and title
%% information and builds the first part of the formatted document.
\maketitle


\secfunc{section function - secfunc} \\
\mention{things to mention / reference - mention} \\
\plan{what to write here - plan} \\
\bp{actual bullet points - bp} \\
\draft{final text drafty - draft} \\
\review{final text review ready - review} \\
\todo{ToDo - todo} \\

\section{Introduction}
\secfunc{why? what? research questions? purpose? hypothesis? wide $\rightarrow$ narrow}

\bp{build break priority for developers \mention{bart paper}
	
decreasing the time until a fix therefore improves overall productivity

most time spend: reading very long and verbose buildlog

recovering relevant information from between many uninteresting other lines

support: extract desired information and present in a structured way

usually: tools use handcrafted regular expressions

but: tedious / mentally complicated work, especially when maintenance is needed some months in the future -> reunderstanding old regular expressions is known to be difficult / dreading task for developers

we: take this burden of developers by enabling them to define extraction programs through simply giving examples of the desired extractions

collect a diverse set of buildlogs to  develop a metamodel of the contained information

implement prototype for pbe extractions in buildlogs using the microsoft PROSE library

showed that our tool is capable of extracting various desired informations, shortens development time in comparison to manual regex construction and performs more accurate than other semi-supervised methods

}

\section{Related Work}

\subsection{Continuous Integration}

\plan{more research into CI now, large scale analyses of buildlogs, open source from travis ci and industrial}

\mention{google build analysis}
\plan{google, java and c++, anlyzed frequency and reasons for failures, few error kinds responsible for most failures, dependency errors appear most, most fixed in two build iterations}

\mention{java os vs ing}
\plan{oss and ing for java projects, classfied failures, cluster analysis, similarities: few compilation \& many testing failures, differences: oss mainly unit testing failures, industrial mainly release preparation}

\mention{travistorrent}
\plan{java and ruby builds \& logs from travis ci, manually built parser to extract failure reasons, most builds fail because of tests, number of tests \& kind \& failure rates very dependent on programming language, low failure rates hint at pretested code on CI server}

\plan{our work: support such research through building log file evaluation easier, enable them to cover more languages and build tool kinds}

\subsection{Tool}

\mention{bart}

\mention{icse paper}



\subsection{Information Extraction and Retrieval}

\mention{PROSE fashextract}

\mention{queriying semi structured data}

\mention{sth about IR / paragraph retrieval}

\plan{difference of us to IR (imprecise \& rough)}

\section{Method}
\secfunc{how? when? material? narrow}

\todo{want?: flow of research figure}

\todo{study design vs. study carry out}

\subsection{Data collection}
\plan{describe process of collecting buildlog files, sampling}

\bp{querying ghtorrent and travis api with (insert fancy name for data collection tool) for x buildlogs of each status of the top-watched repositories on github that also use travis ci}

\subsection{Meta-Model}
\bp{through manual examination of y of the buildlogs we collected designed a metamodell  of the contained/extractable/useful/interesting information}

\plan{figure with metamodel}

\plan{explain (shortly? partly?) meta model classes}

\subsection{Our Tool}
\plan{interaction and usage and output of tool, maybe integration possibilities}

\plan{necessary or interesting implementation details}

\section{Results}
\secfunc{hard numbers! answers to research questions, narrow}

\plan{achieved accuracy, samples needed for same accuracy as related tools (travistorrent), aging of learned programs with newer data}

\section{Analysis / Discussion}
\secfunc{interpretation, implication of answers and why does it matter, comparison to previous findings, narrow $\rightarrow$ wide}

\plan{our tool works well for defined cases, limitations of tool/prose/regex usage in general, every x weeks extraction has to be rewritten to keep accuracy, well improved if only y new examples have to be added}

\section{Conclusion and Future Work}
\secfunc{summarize, results again, future research}

\plan{repeat everything and results}

\plan{future}

\bp{make avaliable to other researchers

integrate into travis

\plan{as always} extend evaluation and data collection

use dataset to evaluate against existing approaches

}

%%
%% The acknowledgments section is defined using the "acks" environment
%% (and NOT an unnumbered section). This ensures the proper
%% identification of the section in the article metadata, and the
%% consistent spelling of the heading.
\begin{acks}
To Robert, for the bagels and explaining CMYK and color spaces.
\end{acks}

%%
%% The next two lines define the bibliography style to be used, and
%% the bibliography file.
\bibliographystyle{ACM-Reference-Format}
\bibliography{paper}

%%
%% If your work has an appendix, this is the place to put it.
\appendix

\section{Research Methods}

\subsection{Part One}


\end{document}
\endinput
%%
%% End of file `sample-sigconf.tex'.
