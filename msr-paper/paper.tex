%%
%% This is file `sample-sigconf.tex',
%% generated with the docstrip utility.
%%
%% The original source files were:
%%
%% samples.dtx  (with options: `sigconf')
%% 
%% IMPORTANT NOTICE:
%% 
%% For the copyright see the source file.
%% 
%% Any modified versions of this file must be renamed
%% with new filenames distinct from sample-sigconf.tex.
%% 
%% For distribution of the original source see the terms
%% for copying and modification in the file samples.dtx.
%% 
%% This generated file may be distributed as long as the
%% original source files, as listed above, are part of the
%% same distribution. (The sources need not necessarily be
%% in the same archive or directory.)
%%
%% The first command in your LaTeX source must be the \documentclass command.
\documentclass[sigconf]{acmart}

% TODO remove XD
\usepackage{xcolor}
\newcommand{\secfunc}[1]{{\color{magenta}#1}}
\newcommand{\mention}[1]{{\color{cyan}#1}}
\newcommand{\plan}[1]{{\color{purple}#1}}
\newcommand{\bp}[1]{{\color{violet}#1}}
\newcommand{\draft}[1]{{\color{blue}#1}}
\newcommand{\review}[1]{{\color{black}#1}}
\newcommand{\todo}[1]{{\color{orange}#1}}
%%
%% \BibTeX command to typeset BibTeX logo in the docs
\AtBeginDocument{%
  \providecommand\BibTeX{{%
    \normalfont B\kern-0.5em{\scshape i\kern-0.25em b}\kern-0.8em\TeX}}}

%% Rights management information.  This information is sent to you
%% when you complete the rights form.  These commands have SAMPLE
%% values in them; it is your responsibility as an author to replace
%% the commands and values with those provided to you when you
%% complete the rights form.
\setcopyright{acmcopyright}
\copyrightyear{2020}
\acmYear{2020}
\acmDOI{10.1145/1122445.1122456}

%% These commands are for a PROCEEDINGS abstract or paper.
\acmConference[Seoul '20]{Seoul '20: Mining Software Repositories Data Showcase}{May 25--26, 2020}{Seoul, South Korea}
\acmBooktitle{Seoul '20: Mining Software Repositories Data Showcase,
  May 25--26, 2020, Seoul, South Korea}
\acmPrice{15.00}
\acmISBN{978-1-4503-9999-9/18/06}


%%
%% Submission ID.
%% Use this when submitting an article to a sponsored event. You'll
%% receive a unique submission ID from the organizers
%% of the event, and this ID should be used as the parameter to this command.
%%\acmSubmissionID{123-A56-BU3}

%%
%% The majority of ACM publications use numbered citations and
%% references.  The command \citestyle{authoryear} switches to the
%% "author year" style.
%%
%% If you are preparing content for an event
%% sponsored by ACM SIGGRAPH, you must use the "author year" style of
%% citations and references.
%% Uncommenting
%% the next command will enable that style.
%%\citestyle{acmauthoryear}

%%
%% end of the preamble, start of the body of the document source.
\begin{document}

%%
%% The "title" command has an optional parameter,
%% allowing the author to define a "short title" to be used in page headers.
\title{Why Did My CI Build Fail?\\ -\\ LogChunks: A Data Set for Build Log Analysis}

\author{Carolin Brandt}
\author{Moritz Beller}
\author{Annibale Panichella}

%%
%% By default, the full list of authors will be used in the page
%% headers. Often, this list is too long, and will overlap
%% other information printed in the page headers. This command allows
%% the author to define a more concise list
%% of authors' names for this purpose.
\renewcommand{\shortauthors}{Brandt, et al.}

%%
%% The abstract is a short summary of the work to be presented in the
%% article.
\begin{abstract}
Build failures are common in continuous integration (CI), but identifying the root cause of a failure is difficult.
Various tools have been proposed to boil down verbose build logs to the specific lines valuable to the developer.
We present \emph{LogChunks}, a collection of 797 Travis CI build logs from 80 GitHub repositories spread over 29 programming languages.
For each build log we manually labeled the log part (chunk) describing why the build failed.
We validated the data set by surveying the developers of the projects which produced the builds.
In addition, the data set categorizes the log chunks according to their format within the log text and provides keywords that a developer would use to search for the log chunks.
\emph{LogChunks} can be the basis to assess further build log analysis techniques and studies about why CI builds fail.



% This is the question developers ask when they want to fix a failing build and the question various researchers ask when they investigate continuous integration practices.
% A common approach is to seek advice in the build's log file, which details the progress, results and errors of the tools involved in the build.
% However, build logs are challenging to process as they are very verbose, at best semi-structured and highly variable between projects.

\end{abstract}

%%
%% The code below is generated by the tool at http://dl.acm.org/ccs.cfm.
%% Please copy and paste the code instead of the example below.
%%
%\begin{CCSXML}
%<ccs2012>
% <concept>
%  <concept_id>10010520.10010553.10010562</concept_id>
%  <concept_desc>Computer systems organization~Embedded systems</concept_desc>
%  <concept_significance>500</concept_significance>
% </concept>
% <concept>
%  <concept_id>10010520.10010575.10010755</concept_id>
%  <concept_desc>Computer systems organization~Redundancy</concept_desc>
%  <concept_significance>300</concept_significance>
% </concept>
% <concept>
%  <concept_id>10010520.10010553.10010554</concept_id>
%  <concept_desc>Computer systems organization~Robotics</concept_desc>
%  <concept_significance>100</concept_significance>
% </concept>
% <concept>
%  <concept_id>10003033.10003083.10003095</concept_id>
%  <concept_desc>Networks~Network reliability</concept_desc>
%  <concept_significance>100</concept_significance>
% </concept>
%</ccs2012>
%\end{CCSXML}
%
%\ccsdesc[500]{Computer systems organization~Embedded systems}
%\ccsdesc[300]{Computer systems organization~Redundancy}
%\ccsdesc{Computer systems organization~Robotics}
%\ccsdesc[100]{Networks~Network reliability}

%%
%% Keywords. The author(s) should pick words that accurately describe
%% the work being presented. Separate the keywords with commas.
\keywords{ci, build log analysis, build failure, chunk retrieval}

%%
%% This command processes the author and affiliation and title
%% information and builds the first part of the formatted document.
\maketitle

\section{Introduction}
+ motivating: why is this data set important/special, what did other people do?

Annibale proposed: what is the problem we are addressing?

\section{Data Schema and Access}
data schema explanation: log, reason the build failed, keywords, structural categories

how to obtain and use (import) the data (github repo, csv?, database dump?)

\section{Data Collection Process}
repo / build / log sampling

labeling process

\section{Data Set Validation}
sending mails to developers

second labeling?

\section{Applications}
\subsection{Chunk Retrieval Evaluation}
short description of our study and results

\subsection{Future Research Questions}
\begin{itemize}
  \item train information chunk retrieval techniques
  \item train build log classification algorithms
  \item look into what keywords tools use to mark failures / developers use to search for them
\end{itemize}

\section{Limitations}
limits: only one coherent text substring labeled, first description of build failure

\section{Potential Improvements}
extension: include more repositories, further classification of the build failures e.g.\ root cause, further validation

%%
%% The acknowledgments section is defined using the "acks" environment
%% (and NOT an unnumbered section). This ensures the proper
%% identification of the section in the article metadata, and the
%% consistent spelling of the heading.

% \begin{acks}
% To Robert, for the bagels and explaining CMYK and color spaces.
% \end{acks}

%%
%% The next two lines define the bibliography style to be used, and
%% the bibliography file.
\bibliographystyle{ACM-Reference-Format}
\bibliography{paper}

%%
%% If your work has an appendix, this is the place to put it.
% \appendix

% \section{Research Methods}

% \subsection{Part One}


\end{document}
\endinput
%%
%% End of file `sample-sigconf.tex'.
