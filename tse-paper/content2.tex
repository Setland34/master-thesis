CI important and widespreach

buildlogs valuable source of information, for devs and researchers

use for many applications -> msr mining chanllenge

though verbose and large -> inadequate for direct consumption

need to retrieve the parts of the logs relevant to 'user', introduce chunks work

-> tools used to retrieve log chunks

however! tedios to develop -> small range of languages spanned

This paper presents a novel technique to retrieve chunks from build logs based
on programming by example. In addition we design and execute an empirical study
on the \emph{LogChunks} data set
to gauge the performance of the novel technique (PBE).
We aim to compare PBE against a common text
similariy measure and keyword search. \todo{eeeeehm}
\todo{difficulty: quite different techniques which are employed to solve a similar
problem -> our contribution: a design and set of metrics to evaluate their performance}

This paper starts by detailing how we applied Programming by Example to the
area of chunk retrieval from build logs using the Microsoft PROSE library.
We present our learnings about the
training input, what you can expect from the output and for which use cases you
could apply Programming by Example in your own field.
Following we illustrate the design and results of our empirical comparison study.
We discuss the results and the methodological obstacles we were found when
trying to compare very diverse techniques addressing the same task.

\textbf{Our work contributes:}
\begin{itemize}
  \item 
\end{itemize}

\todo{RQs?}

\section{Retrieving Chunks from Logs with Programming by Example}
\todo{show and explain examples from defense talk}
Input: in/output examples
then: synthesis of regex program
output: regex program

\subsection{Wins}

\subsection{Caveats and Downsides}

\subsection{Applying Prgramming by Example to other areas}

\section{Empirical Comparsion}
