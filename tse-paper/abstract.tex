Build logs are a textual by-product that software build processes
create, often as part of their Continuous Integration (CI)
pipeline. Build logs can serve as a paramount source of
information for developers to understand and debugg build or
test failures. Manually
extracting the important chunks of information, though, is akin to
finding a needle in a haystack. Recently, researchers and practitioners have begun attempts to partly automate
this time-consuming activity, with the proposition to not only ease developers' tasks but to enable a new class of automated on-ward processing applications.
In this paper, we want to give a systematic overview of the state of the art of the emerging field of build log analysis. To this end, we first survey and categorize the existing methods to extract information chunks from build logs. We then develop prototypical implementations for three promising techniques, namely program synthesis
by example, textual similarity and search keywords. Finally, we evaluate the three techniques in an empirical study on our manually
labeled \emph{LogChunks} data set, which comprises 797 build logs in 29 languages.
Our findings show that none of the three techniques in general outperforms
the others, but rather, that each technique has its respective strengths and weaknesses. We discuss under which circumstances each technique performs best
and provide a recommendation on when developers or researchers should use which
technique.
